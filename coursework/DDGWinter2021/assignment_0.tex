\title{Assignment 0, Written}
\author{Rif A. Saurous}
\date{\today}

\documentclass[11pt]{article}
\usepackage{fullpage}
\usepackage[colorlinks=true]{hyperref}
\usepackage{mathtools}
\usepackage{amsmath}

\newenvironment{exercise}[2][Exercise]{\begin{trivlist}
\item[\hskip \labelsep {\bfseries #1}\hskip \labelsep {\bfseries #2}]}{\end{trivlist}}

\begin{document}
\maketitle

\begin{exercise}{2.1, Euler's Polyhedral Formula.} We want to show that for any polygonal disk,
  $$V - E + F = 1.$$
  The proof is by induction. The base case(s) are a loop of $V$ vertices, $E$ edges, and a single face; the formula holds trivially in this case.

  For the induction step, we'll proceed semi-informally (we'd formalize it as a double-induction on $V$ and $E$ simultaneously), and consider four cases:
  \begin{itemize}
  \item{Removing an interior edge: $V$ remains constant and $E$ and $F$ are both decreased by 1.}
  \item{Removing an interior vertex of degree $d$ and its associated edges: $V$ is decreased by 1, $E$ is decreased by $d$, and $F$ is decreased by $d-1$.}
  \item{Replacing a vertex of degree 2 and its two edges with a single edge: $V$ and $E$ each decrease by 1, $F$ remains constant.}
  \item{Removing an exterior vertex of degree $d$ and its associated edges: again $V$ is decreased by 1, $E$ is decreased by $d$, and $F$ is decreased by $d-1$.}
  \end{itemize}

  In all cases, as we move from more complex to less complex structures, the invariant is maintained.
\end{exercise}

\begin{exercise}{2.2, Platonic Solids.}
  By the Euler-Poincar\'{e} formula,
  $$V - E + F = 2.$$
  Suppose each face has $f$ edges on its boundary. Since each edge appears on two boundaries, we have $fF = 2E$. Similarly, if each edge has degree $d$, we have $dV = 2E$. Substituting,
  $$\frac{2E}{d} - E + \frac{2E}{f} = 2.$$
  Dividing both side by $2E$ and rearranging,
  $$\frac{1}{d} + \frac{1}{f} = \frac{1}{2} + \frac{1}{E}$$,
  and noting that we must have $E > 0$, we see that $d$ and $f$ must satisfy
  $$\frac{1}{d} + \frac{1}{f} > \frac{1}{2}$$.
  Adding in the obvious but not obvious-how-to-justify requirements that $d$ and $f$ are both at least 3, we enumerate all possible solutions (and the resulting platonic solid, although we don't show it's constructable):
  \begin{itemize}
  \item{$d = f = 3$ (tetrahedron)}
  \item{$d = 4, f = 3$, (octahedron)}
  \item{$d = 3, f = 4$, (cube)}
  \item{$d = 5, f = 3$, (icosahedron)}
  \item{$d = 3, f = 5$, (dodecahedron)}
  \end{itemize}
  We note that once $d$ and $f$ are specified, we can easily obtain $V$, $E$ and $F$ through substitution into Euler--Poincar\'{e}. We also note that we've proved these are the only possibilities, but we haven't actually shown the solids are constructable.
\end{exercise}

\begin{exercise}{2.3, Regular Valence.}
  A simplicial surface by definition has 3 edges per face, and if all vertices are regular, each vertex has degree 6, so the surface satisfies $2E = 6V = 3F$. Substituting into Euler-Poincar\'{e},
  $$\frac{E}{3} - E + \frac{2E}{3} = 2 - 2g.$$
  The three terms on the left side sum to zero for any $E$, so any possible solution must set $g = 1$ to force the right side to zero.
\end{exercise}

\begin{exercise}{2.4, Minimum Irregular Valence.}
  Remembering that our simplifical surface by definition satisfies $F = \frac{2E}{3}$, we immediately simplify Euler-Poincar\'{e} to
  $$V - \frac{E}{3} = 2 - 2g.$$
  Next, defining $R$ and $I$ to be respectively the number of regular (degree 6) and irregular (having average degree $d_i$) nodes, we have:
  \begin{eqnarray*}
    V & = & R + I \\
    2E & = & 6R + d_iI \\
  \end{eqnarray*},
  where the latter can be easily manipulated into $\frac{E}{3} = R + \frac{d_i}{6} I$, yielding
  $$V - \frac{E}{3} = (R + I) - (R + \frac{d_i}{6} I) = (1 - \frac{d_i}{6})I = 2 - 2g$$,
  which lets us express $I$ (assuming $d_i \neq 6$) as
  $$I = \frac{12(1-g)}{6-d_i}.$$

  We now turn specifically to the $g=0$ case:
  $$I = \frac{12}{6 - d_i}.$$
  We see that $I$ is increasing in $d_i$, so to minimize $I$, we choose $d_i$ as small as possible. However, {\em every} vertex must have degree at least 3, so $d_i$, the average degree of the irregular vertices, must be at least 3. Therefore the minimal $I$ is obtained by choosing $d_i = 3$, which yields $I=4.$ (In this case, the resulting surface is a tetrahedron.)

  The $g=1$ case was covered in the previous exercise.

  For $g > 1$, we have:
  $$I = \frac{-12k}{6-d_i}$$
  for some positive integer k. To make $I$ a positive integer, we must choose $d_i = 6$. By choosing $d_i = 12k + 6$, we can (according to Euler-Poincar\'{e}, I don't have an explicit construction) achieve $I = 1$.
\end{exercise}

\begin{exercise}{2.5, Mean Valence (Triangle Mesh).}
  As we've explored in the last two exercises, for a triangle mesh, Euler-Poincar\'{e} simplifies to $V - \frac{E}{3} = c$ for some constant $c$ depending on the genus. It's easy to see this implies that as $V \rightarrow \infty, E \rightarrow 3V$. Since the valence $d$ satisfies $d V = 2E$, this implies $d \rightarrow 6$. More prosaically, each time we add a vertex, we are adding 3 edges and 2 faces (we're splitting one face into 3.
\end{exercise}

\begin{exercise}{2.6, Mean Valence (Quad Mesh).}
  For a quad mesh, we have $2E = 4F$, so Euler-Poincar\'{e} simplifies to $V - \frac{E}{2} = c$. Therefore, as $V \rightarrow \infty, E \rightarrow 2V$, and $d \rightarrow 4$. Thinking about manually increasing $V$, I don't see how to add a single vertex $V$, but it's easy to see how to add a new quad ``inside'' an existing quad, which would add 4 vertices, 8 edges, and and 4 faces, implying a limiting ratio of $1:2:1$.
\end{exercise}

\begin{exercise}{2.7, Mean Valence (Tetrahedral).}
  By the ``argument from internal placing'', we obtain a ratio of $1:4:6:3$. This doesn't seem to match the chart at all.

  If we instead take the advice from the article that the link of every vertex is an icosahedron, then each vertex has degree 12, which means that $V:E = 1:6$, each vertex is part of 20 tetrahedra each of which is touched by four vertices, so $V:T = 1:5$, and each tetreaheron has 4 faces each of which is in two tetrahedra, so $F:T = 2:1$; overall this gives us $V:E:F:T = 1:6:10:5$, which seems roughly right. But I'm highly unclear why the ink of every vertex should be an icosahedron.
\end{exercise}

\begin{exercise}{2.8, Star, Closure and Link.}
\end{exercise}

\begin{exercise}{2.9, Boundary and Interior.}
\end{exercise}

\begin{exercise}{2.10, Surface as Permutation.}
\end{exercise}

\begin{exercise}{2.11, Permutation as Surface.}
\end{exercise}

\begin{exercise}{2.12, Surface as Matrices.}
\end{exercise}

\begin{exercise}{2.13, Classification of Simplicial 1-Manifolds.}
\end{exercise}

\begin{exercise}{2.14, Boundary Loops.}
\end{exercise}

\begin{exercise}{2.15, Boundary Has No Boundary.}
\end{exercise}

\end{document}
